\documentclass[a4,english,11pt]{article}

\usepackage{epsfig}
\usepackage{graphics}
\usepackage{graphicx}
\usepackage{amssymb}
\usepackage{t1enc}
\usepackage[latin1]{inputenc}
\usepackage[french]{babel}
\usepackage{fancyhdr}
\usepackage{pstricks,pst-char,pst-coil,pst-fill,pst-plot}

\oddsidemargin-0.5cm \topmargin-1cm \textwidth17cm \textheight25cm
\evensidemargin0mm

\begin{document}

\begin{center}
\begin{huge}
\textbf{Longitudinal Time-Of-Flight taking into account the slope.
February 23 2006}
\end{huge}
\end{center}

After implementing the evaporation method using the ceramic pieces
(started on January 13 2006 lab-book page 147), it was natural to
come back to the problem of the longitudinal TOF that was
investigated previously (June 18 2004 lab-book p128). Indeed the
ceramic is a good way to stop the atomic beam, playing the role of
the pushing laser beam that was used in 2004. This method suffered
the fact that the pushing laser beam was not very local (because
of reflections on the copper and glass tubes), and it was not
clear that at release-time, the spatial distribution was an
heaviside function.

Here, we use a transverse magnetic field to deflect the trajectory
of the atomic beam. A



\end{document}
