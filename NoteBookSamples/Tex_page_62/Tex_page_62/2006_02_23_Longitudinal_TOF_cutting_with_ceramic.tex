\documentclass[a4,english,11pt]{article}

\usepackage{epsfig}
\usepackage{graphics}
\usepackage{graphicx}
\usepackage{amssymb}
\usepackage{t1enc}
\usepackage[latin1]{inputenc}
\usepackage[french]{babel}
%\usepackage{fancyhdr}
%\usepackage{pstricks,pst-char,pst-coil,pst-fill,pst-plot}

\oddsidemargin-0.5cm \topmargin-1cm \textwidth17cm \textheight25cm
\evensidemargin0mm

\begin{document}

\begin{center}
\begin{huge}
\textbf{Longitudinal Time-Of-Flight taking into account the slope.
February 23 2006}
\end{huge}
\end{center}

After implementing the evaporation method using the ceramic pieces
(started on January 13 2006 lab-book page 147), it was natural to
come back to the problem of the longitudinal TOF that was
investigated previously (June 18 2004 lab-book p128). Indeed the
ceramic is a good way to stop the atomic beam, playing the role of
the pushing laser beam that was used in 2004. This method suffered
the fact that the pushing laser beam was not very local (because of
reflections on the copper and glass tubes), and it was not clear
that at release-time, the spatial distribution was an heaviside
function of space (all the atoms before the pushing beam, and none
after).

Here, we use a transverse magnetic field to deflect the trajectory
of the atomic beam so that it hits a ceramic in $z=0$. At $t=0$, we
stop the transverse magnetic field and leave the atoms propagate
towards a probe located in $z=d$, downstream. We take into account a
possible slope like the one implemented on the first $1,7m$ of the
guide ($\alpha=0.012$).

We suppose that at $t=0$, there is no atoms between $z=0$ (the
ceramic) and $z=d$ (the probe). We suppose also that there are no
collisions during the TOF, leading to a simple relation between
the initial velocity distribution at $t=0$ in $z=0$, and the
spatial distribution at a time $t>0$ and $0<z<d$.\\

At the level of the ceramic ($z=0$), at all time, the flux is
$\phi_0$, and the velocity distribution is:

\begin{equation}\label{Velocity}
    P(v) = \sqrt {{\frac {m}{2\, \pi\, k\, T }}}{e^{{\frac {m\, \left( {\it v_0}-v \right) ^{2}}{2\,k\,T}}}}
\end{equation}

But careful: $P(v)$ is the probability for a given atom at the level
of the ceramic to have a longitudinal velocity $v$. If we look at
this probability for a given atom going through the ceramic, this is
different, as for identical densities, faster atoms go through the
ceramic more often than slower atoms.

For a given atom going through, the velocity distribution is
proportional to $v P(v)$:

\begin{equation}\label{F_Velocity}
    F(v) = \frac{v}{v_0}\, \sqrt {{\frac {m}{2\, \pi\, k\, T }}}{e^{{\frac {m\, \left( {\it v_0}-v \right) ^{2}}{2\,k\,T}}}}
\end{equation}

An atom leaving $z=0$ with a velocity $v$ will reach the probe
($z=d$), in a time:

\begin{equation}\label{Td}
    T_d = {\frac {v-\sqrt {{v}^{2}-2\,g\,{\it \alpha}\,d}}{g\,{\it \alpha}}}
\end{equation}

and with a final velocity:

\begin{equation}
    V_d = \sqrt {{v}^{2}-2\,g\,{\it \alpha}\,d}
\end{equation}

We have to notice that $T_d$ has a maximum value ${T_d}_{\rm max} =
\sqrt{\frac{2\,d}{\alpha\,g}}$, meaning that after a time $t =
{T_d}_{\rm max}$, the flux in $z = d$ will be stationary.

The initial velocity distribution $F(v)$ can be expressed in term of
a time-of-arrival distribution:

\begin{eqnarray}\label{F_Td}
  (\ref{Td})\, \Leftrightarrow\, v &=& \frac{d}{T_d} + \frac{g\,\alpha\,T_d}{2} \\
  F(T_d) &=& \frac{\frac{d}{T_d} + \frac{g\,\alpha\,T_d}{2}}{v_0}\, \sqrt {{\frac {m}{2\, \pi\, k\, T }}}{e^{{\frac {m\, \left( {\it v_0}-\frac{d}{T_d} + \frac{g\,\alpha\,T_d}{2} \right) ^{2}}{2\,k\,T}}}}
\end{eqnarray}











\end{document}
