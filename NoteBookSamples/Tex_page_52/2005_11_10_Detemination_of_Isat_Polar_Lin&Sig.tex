\documentclass[a4,english,11pt]{article}


\usepackage{epsfig}
\usepackage{graphics}
\usepackage{graphicx}
\usepackage{amssymb}
\usepackage{t1enc}
\usepackage[latin1]{inputenc}
\usepackage[french]{babel}
\usepackage{fancyhdr}
\usepackage{pstricks,pst-char,pst-coil,pst-fill,pst-plot}


%\markright{} \topmargin0cm \oddsidemargin0cm \textwidth16cm
%\textheight22cm


\oddsidemargin-0.5cm \topmargin-1cm \textwidth17cm \textheight25cm
\evensidemargin0mm


\begin{document}


\begin{center}
\begin{huge}
\textbf{Determination of the effective saturation intensity
$I_{\rm sat}$}
\end{huge}
\end{center}
\ \\

To image the cloud of atoms by absorption, we shine a pulse of
resonant light with an intensity $I_0$. After its propagation
through the packet of atoms, the intensity of the imaging beam
$I(x,y)$ obeys the following equation:
\begin{equation}\label{eq : evo Intensite}
     \frac{I_0}{I_{\rm sat}} - \frac{I}{I_{\rm sat}} + \ln\left(\frac{I_0}{I}\right) =
     DO_{\rm real}\equiv d(x,y)
\end{equation}
where $d(x,y)$ is the real optical density. In the following, $d$
denotes the maximum optical density $d=\max \{d(x,y)\}$.
Experimentally, we measure an effective maximal optical density
$\delta$:
\begin{equation}\label{eq : DO mesur�e}
     \ln\left(\frac{I_0}{I}\right) = DO_{\rm measured}\equiv \delta
\end{equation}
We deduce from Eq. (\ref{eq : evo Intensite}) the equation
fulfilled by the parameters $d$, $\delta$, $I_0$, and $I_{\rm
sat}$:
\begin{equation}\label{eq : relation d delta Isat I0}
    1 - e^{-\delta} = \frac{I_{\rm sat}}{I_0} (d - \delta)
\end{equation}

We now take a series of absorption images of the same cloud but
with different durations $\Delta T$ for the pulse of light while
keeping the product $\Delta T I_0$ constant. $\Delta T$ is varied
from $\Delta T_{\rm min}=250\;\mbox{ns}$ to $\Delta T_{\rm max} =
100\;\mu \mbox{s}$, and the intensity from ${I_0}_{\rm max} \simeq
23\; \mbox{mW/cm}�$ to ${I_0}_{\rm min} \simeq
0.06\;\mbox{mW/cm}�$. The incident intensity of the beam $I_0$ is
measured from the total power of the beam and the size of the beam
at the level of the cloud: $I_0=P/(2\pi\Delta x \Delta y)$ with $P
= 13$ mW, $\Delta x = \ 0.24$ cm and $\Delta_y = 0.37$ cm. From
each image, we can plot the value of the function $g(\delta)$ as a
function of the duration of the pulse $\Delta T$:
\begin{equation}\label{eq : plot}
    g(\delta)=\frac{1 - e^{-\delta}} {d - \delta} =
    \frac{I_{\rm sat}}{I_0} = \frac{I_{\rm sat}}{{I_0}_{\rm max} \Delta T_{\rm min}}
    \Delta T
\end{equation}
Actually, $d$ is obtained in the low incident intensity limit
(i.e. for the "long pulses": $\Delta T \geq 20\;�\mbox{s}
\Rightarrow I_0 \leq 0.3\;\mbox{mW/cm}�$) for which $\delta\simeq
d$. We stress that for this kind of experiment, $d$ needs to be
quite low ($d\leq 1$) in order to minimize the correction to
optical density due to the finite linewidth of the imaging laser.
In practice, we used $d \approx 0.7$ to perform those experiments.

The plot obtained from Eq. (\ref{eq : plot}) gives a straight line
with a slope $I_{\rm sat}/({I_0}_{\rm max} \Delta T_{\rm min})$,
from which we deduce the effective intensity of saturation $I_{\rm
sat}$. We use here only the points $\Delta T \leq 2\;�\mbox{s}
\Rightarrow I_0 \geq 3\;\mbox{mW/cm}�$ for which $\delta<d$, the
other points corresponding to longer pulses are used to determine
$d$. These experiments have been done with various polarizations
for the imaging beam\footnote{"2005\_11\_10\_!\_
DATA\_!\_Polar\_and\_Int\_at\_Nbr\_ph\_fix.OPJ" for the row data,
and 2005\_11\_10\_!\_
STAT\_!\_Polar\_and\_Int\_at\_Nbr\_ph\_fix.OPJ" for the result.}.
The results are summarized in the following table:
\begin{equation}\label{eq : Resultat}
    \begin{tabular}{|l|r|r|r|}
      \hline
      % after \\: \hline or \cline{col1-col2} \cline{col3-col4} ...
     $\mbox{Polar}$ & $I_{0}\ [\mbox{mW/cm}�]$ & $I_{\rm sat}\ [\mbox{mW/cm}�]$ & $\mbox{Mean}$ \\
      \hline     \hline
      $\pi\ 150�$ & $23.3$ & $5.2\pm0.2$ & $5.4\pm0.2$ \\
      \cline{1-3}
      $\pi\ 60� $ & $23.3$ & $5.5\pm0.3$ & \\
      \hline
      $\sigma^+ $ & $22.7$ & $3.6\pm0.1$ & $3.55\pm0.1$ \\
      \cline{1-3}
      $\sigma^- $ & $23.3$ & $3.5\pm0.1$ &  \\
      \hline
    \end{tabular}
\end{equation}

For a given incident intensity, we can saturate more the atoms by
using a circularly polarized light.

\end{document}
