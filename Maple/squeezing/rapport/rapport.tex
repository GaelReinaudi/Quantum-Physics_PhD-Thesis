\documentclass[a4paper,10pt]{article}
\usepackage[french]{babel}

\usepackage[dvips]{graphicx}

\begin{document}

\centerline{\Huge{Squeezing de Spins}}
\centerline{\Huge{Nucl\'{e}aires}}

\bigskip
\bigskip
\bigskip
\bigskip
\bigskip
\bigskip
\bigskip

\centerline{\Large{\bf Etude th\'{e}orique de squeezing}}
\medskip
\centerline{\Large{\bf du spin nucl\'eaire de l'h\'{e}lium 3}}
\medskip
\centerline{\Large{\bf dans son \'etat fondamental en utilisant }}
\medskip
\centerline{\Large{\bf les collisions d'\'echange de m\'{e}tastabilit\'{e} }}

\bigskip
\bigskip
\bigskip
\bigskip
\bigskip
\bigskip
\bigskip

\centerline{\Huge{Reinaudi Ga\"{e}l}}

\bigskip
\bigskip
\bigskip
\bigskip

\centerline{\Large{Stage de DEA Laser-Mati\`{e}re, MIP $3^{\mbox{\small{\textit{\`{e}me}}}}$ ann\'{e}e}}
\bigskip
\bigskip
\bigskip
\bigskip

\centerline{\large{Responsable de stage : Alice Sinatra}}
\medskip
\centerline{\large{Laboratoire Kastler Brossel, D\'{e}partement de physique de l'ENS}}
\medskip
\centerline{\large{alice.sinatra@lkb.ens.fr}}

\bigskip
\bigskip
\bigskip
\bigskip
\bigskip
\bigskip


\centerline{\Large{27 Mars - 2 Juillet 2004}}


\newpage

\section{Pr\'{e}sentation du cadre du stage}
Mon stage de DEA s'est effectu\'{e} du 27 mars au 2 juillet au Laboratoire
Kastler-Brossel de l'Ecole Normale Sup\'{e}rieure.

Dans l'\'{e}quipe du groupe ``fluides quantiques'' qui m'a accueilli, Franck Lal\"{o}e et
ma responsable de stage, Alice Sinatra, \'{e}taient int\'{e}ress\'{e}s par l'\'{e}tude
th\'{e}orique du ``\textbf{squeezing du spin nucl\'eaire de l'atome d'h\'{e}lium 3 en
utilisant les collisions d'\'{e}change de m\'{e}tastabilit\'{e} }".

\section{Le squeezing}
Quand un ensemble de $N$ atomes est polaris\'{e} dans une direction
$\overrightarrow{x}$, l'in\'{e}galit\'{e} d'Heisenberg implique une incertitude sur la
mesure des composantes transverses de spin $S_y$ et $S_z$ qui ne commutent pas entre elles.
\begin{figure}[htb]
\begin{center}
$\Delta S_y . \Delta S_z \ge \frac{1}{2} | \langle [ S_y,S_z ] \rangle | =
\frac{1}{2} \langle S_x  \rangle = \frac{N}{4}$
\end{center}
\end{figure}

\noindent
(on se place en unit\'{e}s atomiques : $\hbar=1$).

En absence de corr\'elations entre atomes (coherent spin state), on atteint
l'\'egalit\'e avec : $\Delta S_y = \Delta S_z = \frac{\sqrt{N}}{2}$
Cette incertitude fondamentale de $\frac{\sqrt{N}}{2}$ est le ``bruit quantique
standard", et on peut atteindre cette limite lors de mesures tr\`{e}s pr\'{e}cises. Le
champ \'{e}lectromagn\'{e}tique dans un \'{e}tat coh\'{e}rent poss\`{e}de aussi un bruit quantique
standard, appel\'{e} ``shot noise".

Le squeezing consiste en la r\'{e}duction du bruit d'une composante transverse
du spin, par exemple $S_y$, sous le
seuil du bruit quantique standard. L'in\'{e}galit\'{e} d'Heisenberg implique alors une
augmentation du bruit de l'observable $S_z$ \
\begin{figure}[h!]
\begin{center}
$\Delta S_y = \xi . \frac{\sqrt{N}}{2} < \frac{\sqrt{N}}{2} $\\

$\Delta S_z = \frac{1}{\xi} . \frac{\sqrt{N}}{2} > \frac{\sqrt{N}}{2} $
\end{center}
\end{figure}
Le spin squeezing est d\'{e}j\`{a} beaucoup \'{e}tudi\'{e} pour ses propri\'{e}t\'{e}s int\'{e}ressantes
dans les mesures de pr\'{e}cision en interf\'{e}rom\'{e}trie atomique (horloges), ou encore
dans le stockage d'\'{e}tats quantiques de la lumi\`{e}re (m\'{e}moires).

Cependant, les atomes dont on a squeez\'{e} le spin sont tr\`{e}s
sensibles \`{a} la d\'{e}coh\'{e}rence (collisions, \'{e}mission spontan\'{e}e), et le squeezing
n'y subsiste pas plus d'une demi-milliseconde typiquement. \

\section{Sujet du stage}
Dans la perspective de prot\'{e}ger le squeezing de la d\'{e}coh\'{e}rence, le but initial
de mon stage \'{e}tait d'\'{e}tudier la possibilit\'{e} de squeezer le spin des atomes
d'h\'{e}lium 3 dans leur \'etat fondamental. En effet l'h\'{e}lium 3 dans son \'{e}tat fondamental $2^1S_0$
poss\`{e}de un spin purement nucl\'{e}aire $F = I = \frac{1}{2}$, et une coh\'{e}rence
stock\'{e}e dans ce niveau, au c\oe ur de l'atome, peut subsister pendant des heures.
Par ailleurs, cet atome poss\`{e}de un \'{e}tat m\'{e}tastable $2^3S$ qui peut servir
d'\'{e}tat de d\'{e}part pour des transitions accessibles par laser et qui peut donc
\^{e}tre utilis\'{e} pour cr\'{e}er des corr\'{e}lations entre les spins des
atomes. Les collisions dites d'\'{e}change de m\'{e}tastabilit\'{e}, qui sont couramment
utilis\'{e}es pour transf\'{e}rer l'orientation des spins de l'\'{e}tat m\'{e}tastable \`{a} l'\'{e}tat
fondamental et cr\'{e}er de la polarisation nucl\'{e}aire, pourraient servir de
couplage entre le fondamental et le m\'{e}tastable afin de ``transf\'{e}rer" le
squeezing \`a l'\'etat fondamental de l'atome.

Le sujet du stage \'{e}tait donc une \'{e}tude th\'{e}orique suivie d'une \'{e}tude de
faisabilit\'{e} de l'exp\'{e}rience qui r\'{e}aliserait une m\'{e}moire quantique \`{a} l'aide de
spins nucl\'{e}aires de l'h\'{e}lium 3 dans son \'{e}tat fondamental.


\section{D\'{e}roulement du stage}
Apr\`{e}s avoir lu les articles directement li\'{e}s au sujet, j'ai d\^{u} me familiariser
avec le traitement quantique des champs en cavit\'{e} et avec le th\'{e}or\`{e}me de
r\'{e}gression quantique.

Lors de mon stage, j'ai \'etudi\'e deux sch\'emas possibles pour squeezer les
spins nucl\'eaires :

Le premier est bas\'e sur une exp\'{e}rience de E.S.Polzik : squeezing
de spins collectifs \`{a} partir de champs coh\'{e}rents, par mesure Quantique
Non-Destrucitve (QND). 
Nous en avons adapt\'{e} la th\'{e}orie pour un traitement des
champs en cavit\'{e} (formalisme de l'\'{e}quation pilote), puis, nous y avons ajout\'{e}
l'\'{e}volution due \`{a} l'\'{e}change de m\'{e}tastabilit\'{e}. J'ai eu \`{a} d\'{e}terminer les
\'{e}quations d'\'{e}volution pour les populations et les coh\'{e}rences.

J'ai calcul\'{e} les spectres de corr\'{e}lations ainsi que les corr\'{e}lations \`{a} temps
\'{e}gaux par diff\'{e}rents moyens \textit{\textbf{(Calculation of quantum correlation
spectra using the regression theorem}}, \textit{Eur. Phys. J. D 12, 339-349}).\\

La deuxi\`eme approche a \'{e}t\'{e} \'elabor\'ee, \`{a} partir d'un article de
A. Dantan et M. Pinard (LKB-Jussieu):\textbf{\textit{Quantum-state transfer
between fields and atoms in electromagnetically induced transparency}},
\textit{PRA 69, 043810 (2004)}. Nous avons adapt\'{e} cette \'{e}tude th\'{e}orique \`{a} la
structure de l'atome d'h\'{e}lium, et y avons ajout\'{e} les \'{e}quations r\'{e}gissant
l'\'{e}change de m\'{e}tastabilit\'{e}.
Une m\'{e}thode de r\'{e}solutions semi-classique (\'{e}quations classiques avec forces de
Langevin) nous a permis de calculer les variances \`{a} temps \'{e}gaux.\\

Des r\'{e}unions r\'{e}guli\`{e}res ont ponctu\'{e} le d\'{e}roulement du stage: avec Franck Lal\"{o}e,
au d\'{e}but du stage pour poser les fondements du probl\`{e}me et \`{a} la fin pour faire
le bilan; avec Michel Pinard et Aur\'elien Dantan (LKB-Jussieu), une fois par mois pour
discuter les r\'{e}sultats et les comparer; avec ma responsable de stage,
quotidiennement.

\section{Outils de travail}
J'ai \'{e}t\'{e} amen\'{e} \`{a} programmer un logiciel de calcul formel (Maple) pour calculer
les \'{e}quations impliquant 10 niveaux atomiques (soit 100 populations et
coh\'{e}rences) et deux champs \'{e}lectromagn\'{e}tiques.
J'ai cr\'{e}\'{e}e un programme calculant automatiquement les commutateurs et les
\'{e}volutions hamiltoniennes et dissipatives, ainsi qu'un programme calculant
l'\'{e}volution due \`{a} l'\'{e}change.\\


\section{R\'{e}sultats obtenus}
\subsection{Squeezing par mesure QND}
La m\'{e}thode inspir\'{e}e de l'exp\'{e}rience de E.S.Polzik, a montr\'{e} que squeezer le
fondamental de l'atome d'h\'{e}lium par mesure QND \'{e}tait th\'{e}oriquement possible,
mais dans des conditions inatteignables exp\'{e}rimentalement (n\'ecessitant de stocker un
photon dans une cavit\'{e} pendant une seconde).

\subsection{Squeezing par transfert d'un vide squeez\'{e} aux atomes}
C'est ce qui nous a amen\'{e}s \`{a} nous pencher sur l'autre m\'{e}thode : squeezing
transf\'er\'{e} du champ aux atomes par Transparence Electromagn\'{e}tiquement Induite
(EIT). Cette m\'{e}thode s'av\`{e}re tr\`{e}s prometteuse puisqu'il semble non seulement
possible de squeezer s\'{e}lectivement le m\'{e}tastable ou le fondamental \`{a} partir
d'un champ vide squeez\'{e}, mais aussi, de relire le squeezing stock\'{e} \`{a} un moment
ult\'{e}rieur.\\

\subsection{Perspective d'exp\'{e}rience}
Mon \'equipe d'accueil et l'\'{e}quipe d'optique quantique du LKB-Jussieu sont motiv\'{e}es
par la mise en place de l'exp\'{e}rience li\'{e}e \`{a} cette th\'{e}orie (exp\'{e}rience qui
serait alors mon sujet de th\`{e}se). Il reste \`{a} confirmer la faisabilit\'{e} th\'{e}orique
en incluant dans le mod\`{e}le d'autres effets, n\'{e}glig\'{e}s jusqu'alors,
comme l'effet Doppler pour un ensemble d'atomes chauds, ou encore la
pr\'esence d'un faible champ magn\'etique.
En effet, les fr\'{e}quences
de Larmor du m\'{e}tastable et du fondamental ont un facteur $10^3$ de diff\'{e}rence
et une telle diff\'{e}rence ne favorise pas le couplage fondamental-m\'{e}tastable. Un
r\'{e}sultat tr\`{e}s r\'{e}cent semble montrer que l'exp\'{e}rience est r\'{e}alisable avec un
champ de l'ordre du milliGauss.

\section{Conclusion}
En plus de l'int\'{e}r\^{e}t que le sujet d'\'{e}tude a \'{e}veill\'{e} en moi durant tout le
stage, j'ai eu l'occasion de d\'{e}couvrir une ambiance de travail tr\`{e}s stimulante
au sein d'une \'{e}quipe particuli\`{e}rement agr\'{e}able. J'ai eu l'impression, pendant
ces trois mois, de construire la partie th\'{e}orique n\'{e}cessaire au montage d'une
exp\'{e}rience prometteuse qui deviendrait mon sujet de th\`{e}se.

\end{document}

