\begin{FrenchAbstract}
%
%
Ce manuscrit pr�sente diff�rentes �tudes exp�rimentales qui s'inscrivent dans le cadre d'une recherche dont l'objectif est la r�alisation d'un \sotosay{\lat} continu et intense.
Nous commen�ons par d�crire le \setup qui permet de mettre en \oe uvre le \rpef d'un \jatgm.
Le gain observ�, d'un facteur~$10$, sur la \ddedp est limit� par la dynamique collisionnelle du \j. 

Pour am�liorer les performances du refroidissement, il appara�t n�cessaire de d�velopper de nouvelles techniques exp�rimentales.
Nous d�taillons ainsi une nouvelle m�thode d'�vaporation tr�s efficace, par mise en contact des atomes du jet avec une surface mat�rielle.

Les autres �tudes men�es portent sur la production et la manipulation de \natufs. 
La premi�re consiste � ralentir des \pats par r�flexion sur un \mimamo. 
La seconde permet la capture et le refroidissement d'une succession de \pats dans un \tpIP.
La derni�re technique met en \oe uvre un \pd, produit par un \fl de forte puissance, afin de produire puis de mettre en mouvement des \patufs tr�s denses. 
Nous pr�sentons enfin un nouveau protocole d'imagerie par absorption donnant acc�s � des mesures quantitatives et pr�cises des \nats optiquement �pais que nous produisons.
%
%
\KeyWords{\lat, �vaporation, \jat, \gm, \pat, \mima, D�mon de Maxwell, \tpIP, \pd, transport non-adiabatique, \ipa.}
\end{FrenchAbstract}


\begin{EnglishAbstract}
%
%
This manuscript presents various studies for an experiment aimed at achieving a continuous and intense \sotosay{atom laser}. We start by describing the experimental setup that allows us to implement the forced evaporative cooling of a magnetically guided atomic beam. The observed gain, by a factor of~10, on the phase space density is limited by the collisional dynamics. 

To improve the performance of evaporative cooling, it appears to be necessary to develop new experimental techniques.
We describe a method of evaporation of the beam, by contact with a material surface.

The other studies address the production and manipulation of ultra-cold atomic clouds.
The first consists in the slowing down of atomic packets by reflection on a moving magnetic mirror. The second allows the capture and cooling of a succession of packets in a train of Ioffe-Pritchard's traps.
The last technique relies on a dipolar trap, produced by a powerful laser beam, in order to produce and then to set in motion dense ultra-cold clouds. We finally report on a new absorption imaging protocol which provides to quantitative accurate measurements for the optically thick atomic clouds that we can produce.
%
%
\KeyWords{atom laser, evaporation, atomic beam, magnetic guide, atomic packet, magnetic mirror, Maxwell's Demon, train of Ioffe-Pritchard's traps, dipole trap, non-adiabatic transport, absorption imaging.}
\end{EnglishAbstract}
