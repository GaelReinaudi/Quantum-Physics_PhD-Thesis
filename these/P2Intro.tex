
\DontFrameThisInToc
\chapter*{Introduction}

Comme nous l'avons montr�, la production d'un jet atomique continu dans un \gm repose sur notre capacit� � produire et � lancer de mani�re r�p�t� des nuages d'atomes ultra-froids. Une fois dans le \gm, les paquets se reouvre pour former un jet continu et il est n�cessaire de mettre en oeuvre un processus d'\evap de mani�re � augmenter la densit� dans l'espace des phases du jet. L'efficacit de ce processus est directement li� au nombre de collision moyen que chaque atome va subir lors de sa propagation dans le \gm. On d�duit de cela deux aspects sur lequels nous pouvons nous concentrer afin d'am�liorer notre dispositif exp�rimantal

\begin{enumerate}
	\item Nous concentrer sur la qualit� de production des \pats afin d'am�liorer 
	\begin{itemize}
		\item le taux de colision initial du jet.
		\item la densit� initiale dans l'espace des phases du jet.
	\end{itemize} 
	C'est l'objet de la troisieme partie de ce manuscrit.
	\item Nous int�resser � l'am�liroation des conditions d'\evap du jet dans le \gm:
	\begin{itemize}
		\item en ralentissant le jet atomique afin de b�n�ficier de plus de temps dans le guide, et donc de laisser le temps au atome de subir plus de colisions.
		\item en conservant momentan�ment dans le \gm un pi�geage suivant trois dimension, plus favorale � l'\evap.
		C'est l'objet de cette partie.
\end{itemize}
\end{enumerate}
