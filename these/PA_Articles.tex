\chapter{R�sum�s des articles r�dig�s dans le cadre de la th�se}\label{annexe:Articles}
Dans cette annexe, nous listons les articles scientifiques qui ont �t� r�dig�s dans le cadre de ma th�se. Pour chacun d'eux, le r�sum� en anglais est fourni.

\newcommand{\citearticleintoc}[1]%
%{~\cite{#1}}
{}

\section{Evaporative cooling of a guided rubidium atomic beam\citearticleintoc{LWR05}}\label{art:LWR05}
\noindent
{\sc T.~{Lahaye}}, {\sc Z.~{Wang}}, {\sc G.~{Reinaudi}}, {\sc S.~P. {Rath}},
  {\sc J.~{Dalibard}} et {\sc D.~{Gu{\'e}ry-Odelin}}, \enquote{{Evaporative
  cooling of a guided rubidium atomic beam}}, {\em Phys. Rev. A\/} {\bf 72},
  n\textsuperscript{o}\kern.2em\relax~3, 033411 (2005).
\\
  DOI:~\href{http://dx.doi.org/10.1103/PhysRevA.72.033411}{10.1103/PhysRevA.72.%
033411}.

\vspace{1ex}
R�sum�:
\begin{quote}\enquote{
We report on our recent progress in the manipulation and cooling of a magnetically guided, high-flux beam of \Rb atoms. Typically, \val{7E9}  atoms  per  second propagate in a magnetic guide providing a transverse gradient of \gausspcm{800}, with a temperature $\approx\microK{550}$, at an initial velocity of \cmps{90}. The atoms are subsequently slowed down to $\approx\cmps{60}$ using an upward slope. The relatively high collision rate (\smun{5}) allows us to start forced evaporative cooling of the beam, leading to a reduction of the beam temperature by a factor of \val{4}, and a tenfold increase of the on-axis phase-space density.%
}\end{quote}


\section{Transport of atom packets in a train of Ioffe-Pritchard traps\citearticleintoc{LRW06}}
\noindent 
{\sc T.~{Lahaye}}, {\sc G.~{Reinaudi}}, {\sc Z.~{Wang}}, {\sc A.~{Couvert}} et
  {\sc D.~{Gu{\'e}ry-Odelin}}, \enquote{{Transport of atom packets in a train
  of Ioffe-Pritchard traps}}, {\em Phys. Rev. A\/} {\bf 74},
  n\textsuperscript{o}\kern.2em\relax~3, 033622 (2006).
\\
  DOI:~\href{http://dx.doi.org/10.1103/PhysRevA.74.033622}{10.1103/PhysRevA.74%
.033622}.

\vspace{1ex}
R�sum�:
\begin{quote}\enquote{
We demonstrate transport and evaporative cooling of several atomic clouds in a chain of magnetic Ioffe-Pritchard traps moving at a low speed ($<\mps{1}$). The trapping scheme relies on the use of a magnetic guide for transverse confinement and of magnets fixed on a conveyor belt for longitudinal trapping. This experiment introduces a different approach for parallelizing the production of Bose-Einstein condensates as well as for the realization of a continuous atom laser.%
}\end{quote}

\section{Evaporation of an atomic beam on a material surface\citearticleintoc{RLC06}}
\noindent 
{\sc G.~{Reinaudi}}, {\sc T.~{Lahaye}}, {\sc A.~{Couvert}}, {\sc Z.~{Wang}} et
  {\sc D.~{Gu{\'e}ry-Odelin}}, \enquote{{Evaporation of an atomic beam on a
  material surface}}, {\em Phys. Rev. A\/} {\bf 73},
  n\textsuperscript{o}\kern.2em\relax~3, 035402 (2006).
\\
  DOI:~\href{http://dx.doi.org/10.1103/PhysRevA.73.035402}{10.1103/PhysRevA.73%
.035402}.

\vspace{1ex}
R�sum�:
\begin{quote}\enquote{
We report on the implementation of evaporative cooling of a magnetically guided beam by adsorption on a ceramic surface. We use a transverse magnetic field to shift locally the beam towards the surface, where atoms are selectively evaporated. With a \mm{5}-long ceramic piece, we gain a factor of $\valpm{1.5}{0.2}$ on the phase-space density. Our results are consistent with a \val{100}\% efficiency of this evaporation process. The flexible implementation that we have demonstrated, combined with the very local action of the evaporation zone, makes this method particularly suited for the evaporative cooling of a beam.%
}\end{quote}

\section{A moving magnetic mirror to slow down a bunch of atoms\citearticleintoc{RWC06}}
\noindent
{\sc G.~{Reinaudi}}, {\sc Z.~{Wang}}, {\sc A.~{Couvert}}, {\sc T.~{Lahaye}} et
  {\sc D.~{Gu{\'e}ry-Odelin}}, \enquote{{A moving magnetic mirror to slow down
  a bunch of atoms}}, {\em Eur. Phys. J. D\/} {\bf 40}, 405--410 (2006).
\\
  DOI:~\href{http://dx.doi.org/10.1140/epjd/e2006-00244-6}{10.1140/epjd/e2006-%
00244-6}.

\vspace{1ex}
R�sum�:
\begin{quote}\enquote{
A fast packet of cold atoms is coupled into a magnetic guide and subsequently slowed down by reflection on a magnetic potential barrier (`mirror') moving along the guide. A detailed characterization of the resulting decelerated packet is performed. We show also how this technique can be used to generate a continuous and intense flux of slow, magnetically guided atoms.%
}\end{quote}

\section{Strong saturation absorption imaging of dense clouds of ultracold atoms\citearticleintoc{RLW07}}
\noindent
{\sc G.~{Reinaudi}}, {\sc T.~{Lahaye }}, {\sc Z.~{Wang}} et {\sc
  D.~{Gu�ry-Odelin}}, \enquote{Strong saturation absorption imaging of dense
  clouds of ultracold atoms}, {\em Opt. Lett.\/} {\bf 32},
  n\textsuperscript{o}\kern.2em\relax~21, 3143--3145 (2007).
\\
  DOI:~\href{http://dx.doi.org/10.1364/OL.32.003143}{10.1364/OL.32.003143}.

\vspace{1ex}
R�sum�:
\begin{quote}\enquote{
We report on a far above saturation absorption imaging technique to investigate the characteristics of dense packets of ultracold atoms. The transparency of the cloud is controlled by the incident light intensity as a result of the nonlinear response of the atoms to the probe beam. We detail our experimental procedure to calibrate the imaging system for reliable quantitative measurements and demonstrate the use of this technique to extract the profile and its spatial extent of an optically thick atomic cloud.%
}\end{quote}

\section{Optimal transport of ultracold atoms in the nonadiabatic regime\citearticleintoc{CKR08}}
\noindent
{\sc A.~{Couvert}}, {\sc T.~{Kawalec}}, {\sc G.~{Reinaudi}} et {\sc
  D.~{Gu�ry-Odelin}}, \enquote{{Optimal transport of ultracold atoms in the
  non-adiabatic regime}}, {\em To appear in Europhys. Lett.\/} (2008).

\vspace{1ex}
R�sum�:
\begin{quote}\enquote{
We report the transport of ultracold atoms with optical tweezers in the non-adiabatic regime, \ie on a time scale on the order of the oscillation period. We have found a set of discrete transport durations for which the transport is not accompanied by any excitation of the centre of mass of the cloud. We show that the residual amplitude of oscillation of the dipole mode is given by the Fourier transform of the velocity profile imposed to the trap for the transport. This formalism leads to a simple interpretation of our data and simple methods for optimizing trapped particles displacement in the non-adiabatic regime.%
}\end{quote}

\section{A Maxwell's demon in the generation of an intense and slow guided beam\citearticleintoc{ReG08}}
\noindent
{\sc G.~{Reinaudi}} et {\sc D.~{Guery-Odelin}}, \enquote{{A Maxwell's demon in
  the generation of an intense and slow guided beam}}, {\em ArXiv e-prints\/}
  {\bf 804.2611} (2008).

\vspace{1ex}
R�sum�:
\begin{quote}\enquote{
We analyze quantitatively the generation of a continuous beam of atoms by the periodic injection of individual packets in a guide, followed by their overlapping. We show that slowing the packets using a moving mirror before their overlapping enables an optimal gain on the phase space density of the generated beam. This is interpreted as a Maxwell's demon type strategy as the experimentalist exploits the information on the position and velocity of the center of mass of each packet.%
}\end{quote}

\section{A quasi-monomode guided atom-laser from an all-optical BEC\citearticleintoc{CJK08}}
\noindent
{\sc A.~{Couvert}}, {\sc M.~{Jeppesen}}, {\sc T.~{Kawalec}}, {\sc
  G.~{Reinaudi}}, {\sc R.~{Mathevet}} et {\sc D.~{Gu�ry-Odelin}}, \enquote{{A
  quasi-monomode guided atom-laser from an all-optical Bose-Einstein
  condensate}}, {\em ArXiv e-prints\/} {\bf 802.2601} (2008).

\vspace{1ex}
R�sum�:
\begin{quote}\enquote{
We report the achievement of an optically guided and quasi-monomode atom laser, in all spin projection states ($m_F =$ $-1$, $0$ and $+1$) of $F=1$ in Rubidium 87. The atom laser source is a Bose-Einstein condensate (BEC) in a crossed dipole trap, purified to any one spin projection state by a spin-distillation process applied during the evaporation to BEC. The atom laser is outcoupled by an inhomogenous magnetic field, applied along the waveguide axis. The mean excitation number in the transverse modes is $\langle n \rangle = \valpm{0.65}{0.05}$ for $m_F = 0 $ and $ \langle n \rangle = \valpm{0.8}{0.3}$ for the low field seeker $m_F = -1$.%
}\end{quote}

