
\def\skipInTitle{0.2cm}
\newcommand{\TitreTheseA}{%
Manipulation et \rpef}
\newcommand{\TitreTheseB}{%
d'ensembles atomiques \ufs pour la production d'un}
\newcommand{\TitreTheseC}{%
jet intense dans le régime de dégénérescence quantique :}
\newcommand{\TitreTheseD}{% 
vers l'obtention d'un \textit{laser à atomes} continu}

\thispagestyle{empty}

\leftline{\bfseries \Large Département de physique \hskip 3cm
Laboratoire Kastler Brossel} 
\vskip 0.2cm \leftline{\bfseries \Large
\'Ecole Normale Supérieure}

\begin{center}
\includegraphics[width=35mm]{chouettes}
\end{center}

\centerline{\bfseries \sffamily \large TH\`ESE de DOCTORAT de
l'UNIVERSIT\'E PARIS 6} 
\vskip .5cm 
\centerline{\large Spécialité~:}
\centerline{\large Physique Quantique} 

\vskip 1.4cm

\centerline{\large présentée par} 
\vskip .6cm
\centerline{\bfseries \sffamily \Large Gaël REINAUDI} 
\vskip .6cm
\centerline{\large pour obtenir le grade de DOCTEUR de
l'UNIVERSIT\'E PARIS 6}
 
\vskip 1.2cm

%\centerline{\textit{\large Sujet de la thèse:}}
\vskip .3cm
\centerline{\line(1,0){18}} 
\vskip 0.4cm 
\centerline{\bfseries \sffamily \LARGE \TitreTheseA }
\vskip \skipInTitle 
\centerline{\bfseries \sffamily \LARGE \TitreTheseB }
\vskip \skipInTitle 
\centerline{\bfseries \sffamily \LARGE \TitreTheseC }
\vskip \skipInTitle 
\centerline{\bfseries \sffamily \LARGE \TitreTheseD }
\centerline{\line(1,0){18}}

\vskip 1.5cm

\centerline{\bfseries \centerline{\bfseries  \large Soutenue le 11 Juillet 2008}}

\centerline{\large devant le jury composé de~:} 

\vskip 0.9cm

\begin{center}
\begin{minipage}{14cm}
\large\centering
\begin{tabular}{lp{8cm}l}

%%%%%%%%%%%%%%%%%%%%%%%%%%%%%%%%%%%%%%%%%%%%%%%%%%%%%%%%%%%%%%%%%%%
%%%%%%%%%%%%%%%%%%%%%%%%%%%%%%%%%%%%%%%%%%%%%%%%%%%%%%%%%%%%%%%%%%%
\bf M.&	\ \bf  Antoine BROWAEYS\ \rm\dotfill			&%
Rapporteur%
\vspace{.1cm}\\
\bf M.&	\ \bf  Pierre LEMONDE\ \rm\dotfill			&%
Rapporteur%
\vspace{.1cm}\\
\bf M.&	\ \bf  Jacques TREINER\ \rm\dotfill	&%
Président du jury%
\vspace{.1cm}\\
\bf M.&	\ \bf  Claude COHEN-TANNOUDJI\ \rm\dotfill			&%
Examinateur%
\vspace{.1cm}\\
\bf M.&	\ \bf  Bruno DESRUELLE\ \rm\dotfill			&%
Examinateur%
\vspace{.1cm}\\
\bf M.&	\ \bf David GUÉRY-ODELIN\ \rm\dotfill		&%
Directeur de thèse%
%%%%%%%%%%%%%%%%%%%%%%%%%%%%%%%%%%%%%%%%%%%%%%%%%%%%%%%%%%%%%%%%%%
%%%%%%%%%%%%%%%%%%%%%%%%%%%%%%%%%%%%%%%%%%%%%%%%%%%%%%%%%%%%%%%%%%
%
%\bf M.&	\ \bf  Jean-Pierre Reinaudi\ \rm\dotfill  &Père\vspace{.1cm}\\
%\bf M.&	\ \bf  Christian Monzier\ \rm\dotfill			&Beau-Père\vspace{.1cm}\\
%\bf M.&	\ \bf  Adeline Monzier\ \rm\dotfill				&Président du gnou\vspace{.1cm}\\
%\bf M.&	\ \bf  Adrien Mahé\ \rm\dotfill						&Ami\vspace{.1cm}\\
%\bf M.&	\ \bf  Cedric Roux\ \rm\dotfill						&Ami\vspace{.1cm}\\
%\bf M.&	\ \bf  Lisa Reinaudi-Monzier\ \rm\dotfill	&Bébé \vspace{.1cm}\\


\end{tabular}
\end{minipage}
\end{center}


%-------------------------------------------------------------------
%                          remerciements
%-------------------------------------------------------------------


%\DontFrameThisInToc
\begin{ThesisAcknowledgments}

Le travail de thèse que je présente dans ce manuscrit s'est déroulé au \nomofficiel{Laboratoire Kastler Brossel} de novembre 2004 à juillet 2008. Je remercie les directeurs successifs du laboratoire, Frank Laloë et Paul Indelicato, de m'y avoir accueilli et fait bénéficier des conditions de recherche exceptionnelles qui y règnent.

Je remercie vivement les membres du jury:
Antoine Browaeys et Pierre Lemonde pour avoir accepté le rôle de rapporteurs de ce manuscrit, 
Jacques Treiner pour avoir présidé le jury,
Claude Cohen-Tannoudji pour l'intérêt qu'il a porté à notre expérience durant ces années,
et Bruno Desruelle qui a joué un rôle important dans le financement de ma thèse, et dont l'enthousiasme et la confiance ont toujours été très motivants. 

 
\subsection*{Merci aux membres du laboratoire}

{\AjouteLigne}

%Ce travail de thèse n'aurait bien sur pas été possible sans l'aide et la participation de nombreuses personnes.

Je tiens à remercier tout particulièrement mon directeur de thèse, \dgo, qui a su encadrer mon travail de thèse. 
David est particulièrement attentif au déroulement de la thèse de ses étudiants. Il est toujours disponible afin de de répondre aux questions, d'orienter, de faire part de ses idées, et d'en discuter avec toute l'équipe. Il a de plus une vision à moyen et long termes en ce qui concerne l'avenir de ses thésards.

Outre ses qualités de scientifique, David est quelqu'un avec qui il est agréable de discuter sur divers sujets: cinéma, littérature, actualités, etc... 
Il a le don de savoir placer de petites pointes d'humour tout au long d'une journée bien remplie.
%Il me semble pouvoir dire que chaque jour, autour d'un bon café. 
Le café est une composante importante d'une journée typique dans l'équipe de David.
Quand j'ai commencé ma thèse je ne buvais pas de café, ou tout du moins pas d'\termetech{expresso}. David avait parié que d'ici ma soutenance, je serais \sotosay{converti},... j'avais affirmé que {non}. 
%Je dois bien reconnaître que David avais raison. 
Après avoir tenté (en vain) de m'intégrer à l'équipe sans boire l'un des cinq cafés quotidiens, j'ai tenté l'expérience, d'abord en coupant le café à l'eau froide (David a parfois bien failli jeter l'éponge), puis en sucrant intensément. Après trois ans et demi de thèse, je sais maintenant déguster un bon \termetech{expresso} sans sucre. Donc merci David.


J'ai eu l'immense plaisir de travailler quelques mois avec Alice Sinatra sur un sujet théorique avant le début de ma thèse. C'est grâce à elle que j'ai pu intégrer le Laboratoire Kastler Brossel. La bonne humeur, la sympathie et l'efficacité d'Alice m'ont vraiment beaucoup marqué.
Je remercie aussi Jean Dalibard, Christophe Salomon, et Yvan Castin pour l'intérêt qu'ils ont porté à notre expérience et l'interaction constructive qu'ils ont eu avec notre équipe en de multiples occasions.


Je tiens à remercier chaudement les membres de l'équipe avec lesquels j'ai travaillé durant ces années.
Thierry Lahaye qui était le principal artisan du montage expérimental à mon arrivée dans l'équipe, et qui m'a enseigné les techniques relatives au domaine des atomes-froids. 
Antoine Couvert qui soutiendra sa thèse en 2009 a été un \termetech{cothésard} avec qui il a été très agréable de travailler. Il a le goût du travail bien fait, et de la précision. Impossible d'oublier son fameux \termetech{lapin à la bière sans gluten}.
%, ni cette incroyable bouteille de vin indéboucher.
Matthew Jeppesen, Australien, efficace et sur-motivé, a effectué un stage en cotutelle de six mois dans notre équipe. 
Zhaoying Wang et Tomasz Kawalec ont travaillé comme postdocs sur l'expérience.
Giovanni Luca Gattobigio, efficace et plein d'humour, effectue en ce moment son postdoc dans l'équipe. J'apprécie beaucoup son sens des priorités et de l'organisation du temps.
J'ai aussi eu le plaisir de travailler avec Patrick Rath et Renaud Mathevet.

Je remercie les différents collègues des autres équipes, avec lesquels j'ai interagi.
L'équipe \sotosay{Lithium}: Leticia, Martin, Jason, Nir, Sylvain.
L'équipe \sotosay{Rubidium 1}: Sabine, Baptiste, Marc, Patrick, Tarik.
L'équipe \sotosay{Helium}: Nassim, Christian, Maximilien, Steven.
L'équipe \sotosay{Fermix}: Frédéric, Armix, Thomix.
Je remercie aussi tout le personnel technique et administratif du département de physique.

\subsection*{Merci aux personnes qui m'entourent}
{\AjouteLigne}
Je remercie bien évidemment Adeline pour m'avoir soutenu et pour sa bonne humeur de chaque instant. 
Je dois remercier Lisa pour avoir fait ses nuits dès sa sixième semaine (disons plutôt la plupart de ses nuits).
\`A toutes les deux, je dédie ce manuscrit de thèse.

Une mention spéciale pour Had et Ced qui ont été d'un grand soutien durant ces années, autant au labo que dans la vie de tous les jours, à l'ENS, et après. 
Adrien est le parrain de ma fille Lisa, et je lui dédie la présence fortuite du mot \parole{idoine} dans ce manuscrit.
Cédric et moi avons collaboré pour mettre au point l'aspect graphique de nos thèses respectives%
\footnote{Toutes les figures et encadrés de ce manuscrit ont été faits à partir d'un \termetech{package} latex, \termetech{Tikz}, que je recommande vivement pour son coté intuitif, sa flexibilité et sa compatibilité avec le format \termetech{pdf}.}%
. Had et Ced savent comme moi sous quelle forme se présente un \termetech{disrupteur dimensionnel}%
\footnote{\parole{Le disrupteur dimensionnel est un objet énigmatique très puissant... Et l'énigme qui l'entoure n'a d'égal que son immense puissance.}}%
.
Merci aussi à Camille et Anne-Laure bien-sûr.

Je remercie ma mère, Dominique, dont le soutien a été inestimable tout au long de mes études et de ma vie professionnelle.
Je remercie mon père, Jean-Pierre, à qui je dois sûrement mon goût du raisonnement logique et qui m'a très tôt appris les bases de l'électronique et de la mécanique.
Christian a été, plus qu'un soutien moral, le plus fidèle supporter de mon travail de thèse.
Sa passion pour la Science et le Rationalisme n'a d'égale que sa générosité.
Ses multiples relectures de mon manuscrit m'ont été d'une grande utilité. J'ai aussi beaucoup pensé à Christine durant ces années et j'aurais voulu qu'elle puisse lire cette thèse.

Rédiger la thèse au labo s'est rapidement avéré être une tâche difficile en raison de l'encombrement et de l'animation qui règnent dans les bureaux. 
C'est mon frère Robin qui m'a conseillé de rédiger ma thèse dans un café, et je le remercie pour cette brillante idée qui ne me paraissait, de prime abord, pas idoine.
\RemonteUnPeuFig
\Remarque{
J'ai rédigé la plus grande partie de ce manuscrit dans le café \termetech{La Contrescarpe}, situé sur la place du même nom. L'accès \termetech{WiFi} et la possibilité d'y rester des journées entières m'ont permis de rédiger en toute tranquillité. Merci à Nathalie pour sa bonne humeur quotidienne. Je regrette juste que le sachet de thé soit passé à \val{4.60} euros, alors même que le salaire des serveurs est resté étonnamment stable.
}
\RemonteUnPeuFig

Je remercie chaleureusement mes amis.
Tout d'abord Josselin qui était si enthousiaste à l'idée que je commence une thèse. J'aurais voulu qu'il puisse la lire.
Mes deux \sotosay{potes de prépa},
Juan, qui partage ma passion pour la Science et le Rationnel. J'ai beaucoup appris au contact de ce véritable \termetech{maître Yoda} du \termetech{mojo}; et
Benoit, qui m'avait dit qu'il deviendrait clown dans un cirque itinérant (s'il ne partait pas en finance je suppose donc).
Merci à Céline Temps pour sa confiance en moi.
Merci à Pascal et Romain, infatigables escaladeurs avec lesquels les mots \parole{7$c$ après travail} ont fini par prendre un sens (je ne saurais reproduire notre cri de guerre sur le papier). Bonne chance à Audrey pour la fin de sa thèse.

Merci à Ronan pour ses petits coups de pouce aussi inattendus qu'utiles, et à Céline qui est officiellement la meilleure amie de Lisa. Merci à Aurélien pour son soutien, mais j'attends toujours qu'il m'explique pourquoi mon téléphone portable capte parfaitement bien lorsque je le place dans l'enceinte d'un four micro-onde (éteint).

J'en profite pour remercier Laurent Lepetit, mon prof de physique de \termetech{sup} au Lycée Descartes de Tours, pour avoir contribué à éveiller ma passion pour la physique par son enthousiasme, sa pédagogie et son charisme. C'est sûrement grâce à lui que les années de \termetech{prépa} se sont passées dans une si bonne et si constructive ambiance.




\end{ThesisAcknowledgments}

%-------------------------------------------------------------------
%                            dedicace
%-------------------------------------------------------------------

\begin{ThesisDedication}
\`A Adeline,\\
et à Lisa qui a eu un an le jour de ma soutenance.
\end{ThesisDedication}


