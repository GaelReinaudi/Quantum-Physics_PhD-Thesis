\chapter{Force dipolaire}\label{annexe:ForceDipolaire}
\minitoc

Dans cette annexe nous décrivons la \fd électrique. 
%Nous commencerons par une approche qualitative de cet effet en utilisant un raisonnement classique. 
%Ensuite, nous introduirons 
Nous y décrirons les étapes d'un calcul quantitatif menant à une interprétation en terme de déplacement des niveaux d'énergie de l'atome. Nous appliquerons enfin le résultat au cas d'un atome alcalin tel que le \Rb.

\EnFaitNon{
\subsubsection{Approche qualitative par la mécanique classique}
Pour décrire qualitativement la \fd, nous décrirons ici l'atome par un modèle de Lorentz d'un oscillateur classique%
\footnote{Le modèle de Lorentz consiste à considéré que l'électron est élastiquement lié à l'atome par une force de rappel de type Hooke : $F(x)=-k\,x$.}%
. 
Lorsqu'un atome est placé dans une lumière laser, le \che $\Evect$ y induit un moment dipolaire atomique $\Pvect$ oscillant à la pulsation $\pulsLaser$ de l'onde. En utilisant les notations complexes usuelles, 
\begin{equation}
\begin{cases}
\Evect(\vectr,t) = & \Eampl\,\Expo{-\im\,\pulsLaser} \,\,\vectu \, + \cc \nonumber \\
\Pvect(\vectr,t) = & \Pampl\,\Expo{-\im\,\pulsLaser} \,\,\vectu \, + \cc \nonumber
\end{cases}
%\virguleformule
%	\label{eq:EvectPvect}
\end{equation}
où $\Eampl$ et $\Pampl$ sont respectivement les amplitudes complexes du \che et du \di, et $\cc$ désigne le \termetech{complexe conjugué}.
L'amplitude du \di s'exprime alors par:
\begin{equation}
	\Pampl = \Polarisation \, \Eampl
\virguleformule
%	\label{eq:DipInduit}
\end{equation} 
où $\Polarisation$ est la polarisabilité complexe de l'atome, qui dépend de la pulsation $\pulsLaser$ de l'onde excitatrice.

Le piégeage optique par forces dipolaires repose sur l'interaction entre :
\begin{itemize}
	\item le \che $\Evect$ de l'onde laser,
	\item et le dipôle $\Pvect$ que ce dernier induit dans les atomes.
\end{itemize}

}


\section{Méthode de \termetech{l'atome habillé}}
Pour calculer de manière quantitative l'effet de la force dipolaire sur un atome, il faut en fait tenir compte :
\begin{itemize}
	\item de la structure interne de l'atome, 
	\item de l'occupation des différents sous-niveaux énergétiques,
	\item de la pulsation et polarisation de la lumière.
\end{itemize}
Il faut utiliser pour cela un formalisme adapté à ce traitement. Nous allons employer ici la méthode de \emph{l'atome habillé}~\cite{CDG98}. Celle-ci, au lieu de traiter le système \{\termetech{atome}\} immergé dans un champ extérieur classique, consiste à considérer le système \{\termetech{atome}~+~\termetech{champ quantique associé à un mode particulier du champ}\}. Cette formulation présente l'avantage d'avoir un hamiltonien indépendant du temps et permet donc d'introduire de vrais niveaux d'énergie.

\subsection{Modélisation}
Pour simplifier au maximum la description quantique du champ laser, nous considérons une cavité \emph{sans pertes}, dont \emph{un seul mode}, de fréquence $\pulsLaser$, contient des photons. Il sera par la suite possible de faire un lien entre le nombre $\Nph$ de photons dans la cavité et l'intensité moyenne $I$ de l'onde laser:
\begin{equation}
	I = \hbar \, \pulsLaser \, c \, \frac{\Nph}{\Vcav}
	\virguleformule
	\label{eq:IlaserNphotons}
\end{equation}
où $\Vcav$ est le volume de la cavité. 
Dans la suite, les états propres de l'atome \termetech{habillé} par le champ en l'absence de couplage seront notés $\Ket{\AiN}$ et correspondent au produit tensoriel :
\begin{equation}
\Ket{\AiN} \equiv \Ket{A_i} \otimes \Ket{\Nph}
\virguleformule
	\label{eq:KetAiN}
\end{equation}
où $\Ket{A_i}$ désigne l'état interne de l'atome, $\Ket{\Nph}$ désigne l'état du champ caractérisé par le nombre $\Nph$ de photons dans le mode laser%
%\footnote{Le fait de repéré l'état du champ quantifié par le nombre $\Nph$ de photons qui y participe suppose que nous modélisions le mode de l'onde laser par un mode libre d'une cavité arbitrairement grande et sans pertes~\cite{CDG98}.}%
.
L'énergie d'un état $\Ket{\AiN}$ est la somme de l'énergie interne $\Ener_{Ai}$ de l'atome et de l'énergie $\Nph\,\hbar\,\pulsLaser$ du champ.



\subsection{Couplage atome-laser}
Dans le \termetech{point de vue dipolaire électrique}%
\footnote{Le point de vue dipolaire électrique consiste à faire l'\termetech{approximation dipolaire}, \cad à considérer que l'extension spatiale de l'atome est très inférieure à la \lo du laser. }%
, le couplage $\VAL$ entre l'atome et le mode laser s'écrit :
\begin{equation}
\VAL = - \OperP \cdot \OperE = e\,\OperR\cdot\OperE
\virguleformule
	\label{eq:VAL}
\end{equation}
où $\OperP$ est l'opérateur dipôle atomique,
\begin{equation}
\OperP = - e \, \OperR ~~~~\mbox{($\OperR$ est l'opérateur position et $e$ la charge élémentaire)}
\virguleformule
	\label{eq:OpPos}
\end{equation}
et $\OperE$ est l'opérateur champ laser évalué à la position de l'atome. La forme générale de l'opérateur champ dans la cavité est : 
\begin{equation}
	\OperE = \Polarisation 
	\, \sqrt{\frac{\hbar\,\pulsLaser}{2\,\epsilon_0\,\Vcav}} 
	\, (a + a^\dag)
\virguleformule
	\label{eq:OpChamp}
\end{equation}
où $a$ et $a^\dag$ sont les opérateurs \termetech{annihilation} et \termetech{création} de photons dans le mode considéré, et $\Polarisation$ un vecteur normé à coefficients complexes définissant la polarisation de l'onde laser.

\subsection{Interprétation des termes du hamiltonien}\label{sec:InterpretationCouplage}
Dans cette sous-section, nous nous proposons d'interpréter brièvement les termes du couplage $\VAL$ en considérant leur effet sur les état propres $\Ket{\AiN}$. Pour cela, nous écrivons $\VAL$ sous la forme:
\begin{equation}
	\VAL = 
	\sqrt{\frac{\hbar\,\pulsLaser\,e^2}{2\,\epsilon_0\,\Vcav}} 
	\, \Bigl( a + a^\dag \Bigr)
	\, \Bigl( \Polarisation \cdot \OperR \Bigr)
	\pointformule
	\label{eq:VALInterpretation}
\end{equation}
Cette expression permet de dégager les points suivants quant au couplage de deux états propres $\Ket{\AiN}$:
\begin{ditemize}
	\item le terme dans la première parenthèse%
	\footnote{Rappelons comment $a$ et $a^\dag$ agissent sur un état du champ $\Ket{N}$ :\\ $a\,\Ket{N} = \sqrt{N}\,\Ket{N-1}$ et $a^\dag\,\Ket{N} = \sqrt{N+1}\,\Ket{N+1}$.}%
	, $a + a^\dag$, n'agit que sur la partie champ $\Ket{N}$ et ne peut coupler $\Ket{\AiN}$ qu'avec des états ayant un nombre de photons $\Nph' = \Nph \pm 1$.
%	:
%	\[
%	\Ket{\AiN} \xrightarrow[\VAL]{\phantom{aaaaaaaaa}} \left\{ \KetAjNpm \right.
%	\pointformule
%	\]
	\item le terme dans la seconde parenthèse, $\Polarisation \cdot \OperR$, n'agit que sur la partie atomique. Il définit, en fonction de la polarisation, les règles de sélection et les probabilités des transitions atomiques.
\end{ditemize}
L'opérateur $\VAL$ couplera a priori chaque état $\Ket{\AiN}$ aux états $\Ket{A_j, \Nph+1}$ et $\Ket{A_j, \Nph-1}$.

\casse

\section{Traitement perturbatif du couplage}
Pour trouver les états propres du système \{\termetech{atome} + \termetech{champ} + \termetech{couplage}\}, il faudrait diagonaliser son hamiltonien. En pratique, nous allons éviter cette tâche fastidieuse et traiterons l'effet du couplage $\VAL$ de manière perturbative : nous allons considérer que les états propres $\Ket{\AiN}$ du système \{\termetech{atome} + \termetech{champ}\} restent des états propres en présence du couplage $\VAL$, mais avec une petite modification $\Delta \Ener_i$ de leur énergie:
\[
\Ener_i \xrightarrow[\VAL]{\phantom{aaaaaaaaa}} \Ener_i + \Delta\Ener_i
\pointformule
\]
\Remarque%
{
Pour justifier ce traitement, nous devons d'abord nous assurer que l'énergie typique $\EAL$ mise en jeu par le couplage $\VAL$ est très faible devant toutes les autres échelles d'énergie présentes dans le hamiltonien sans interaction. 

Notamment, $\EAL$ doit être très faible devant les termes de structure hyperfine : 
$\Delta \Ener_i \ll \hbar\,\omegaHF$, 
où $\omegaHF$ est la pulsation entre les niveaux hyperfins. 
}
Les équations~\nref{eq:VAL} à~\nref{eq:OpChamp} permettent de calculer un ordre de grandeur de l'énergie typique $\EAL$ mise en jeu par le couplage $\VAL$. En effet, l'opérateur position $\OperR$ correspond à une distance de l'ordre de, ou inférieure, au rayon de l'atome ($\approx a_0$, le rayon de Bohr%
\footnote{Rappelons que le rayon de Bohr vaut $a_0\approx \SI{0.5}{\angstrom}$.}%
), et les opérateurs $a$ et $a^\dag$ ont pour valeur moyenne typique $\sqrt{\Nph}$. On a donc :
\[
\EAL \approx e\,a_0\,\sqrt{\frac{\Nph\,\hbar\,\pulsLaser}{2\,\epsilon_0\,\Vcav}}
 = e\,a_0\,\sqrt{\frac{I}{2\,\epsilon_0\,c}}
\virguleformule
\]
où la deuxième égalité fait intervenir l'expression~\nref{eq:IlaserNphotons}. Avant de continuer ce calcul d'ordre de grandeur, nous devons mentionner un point important, à savoir que, pour des raisons de symétrie des fonctions d'onde de l'électron, l'opérateur $\VAL$ a une moyenne rigoureusement nulle pour tout état propre $\Ket{A_i}$, et par conséquent : 
\[
\BraOpKet{\AjN}{\VAL}{\AiN} = 0
\pointformule
\]
Le premier terme du développement perturbatif à prendre en compte est donc, a priori, le terme d'ordre $2$. D'après l'équation \vref{eq:EiModiff}, l'ordre de grandeur de ce terme est $\ttfrac{\EAL^2}{\EcartMini}$, où $\EcartMini$ est l'écart énergétique typique%
\footnote{En réalité, nous voulons estimer une valeur maximale de $\Delta\Ener_i$ par l'équation~\nref{eq:EiModiff}. L'écart d'énergie $\EcartMini$ à considérer est donc l'écart \emph{minimal} entre deux états couplés par $\VAL$.}
entre les états de l'atome \termetech{habillé} couplés par $\VAL$.
%
\ApplicationNumerique{
La condition $\Delta \Ener_i \ll \hbar\,\omegaHF$ qui justifie le traitement perturbatif de l'effet de $\VAL$ peut donc se traduire par:
\begin{equation}
	\frac{\EAL^2}{\EcartMini} \approx \frac{I}{\EcartMini}\,\frac{e^2\,{a_0}^2}{2\,\epsilon_0\,c} \ll  \hbar\,\omegaHF
\pointformule
	\label{eq:ConditionPerturbatif}
\end{equation}
Dans nos conditions expérimentales, la \lo du \lyb est \mbox{$\LO\approx\micron{1}$} et les transitions à considérer pour le \Rb sont les lignes $D_1$ ($\nm{795}$) et $D_2$ ($\nm{780}$). On a donc $\EcartMini\approx\hbar\times\SI{6E14}{\reciprocal\second}$.
Dans l'état fondamental du \Rb, on a $\omegaHF \approx \SI{4E10}{\reciprocal\second}$.
Finalement on aboutit à :
\begin{equation}
	I \ll \SI{5E12}{\watt\per\square\meter}
\virguleformule
	\label{eq:ConditionIntPerturbatif}
\end{equation}
ce qui est une intensité extrêmement élevée puisqu'elle correspond, par exemple%
\footnotemark%
, à un \fl de rayon \micron{50} et d'une puissance de \W{20000}.
Nous serons donc bien dans des conditions satisfaisantes pour utiliser un traitement perturbatif. 
}
\footnotetext{Un autre exemple serait celui d'un \fl de \W{300}, focalisé sur un disque de rayon \micron{5}.}



\subsection{Calcul du déplacement lumineux}
Le déplacement lumineux sera calculé en appliquant la théorie des perturbations au second ordre d'états non-dégénérés~\cite{CDL86b}:
\begin{equation}
	\Delta\Ener_i = 
	\sum_{\substack{j \neq i \\ \Nph'}} 
	\dfrac{
	\Module{\BraOpKet{\AjN}{\VAL}{\AiN}}^2
	}{\EneriN - \EnerjN}
	\pointformule
	\label{eq:EiModiff}
\end{equation}
Rappelons que $\EneriN$ et $\EnerjN$ sont les énergies des états propres de l'atome \termetech{habillé} en l'absence de couplage, \cad : $\EneriN = \Ener_i +\Nph\,\hbar\pulsLaser$.
%
\Remarque{\label{RQ:QuePolarLinEtCir}
Comme nous venons de le dire, l'équation~\nref{eq:EiModiff} s'applique au cas de niveaux non-dégénérés. Or, chaque niveau de structure hyperfine de l'atome de \Rb est composé de $2\,F+1$ \snx dégénérés (en l'absence de champ magnétique), avec $F$ le moment angulaire du niveau considéré. on peut toutefois appliquer le développement~\nref{eq:EiModiff} à condition qu'il n'y ait aucun couplage entre ces \snx. En pratique, du fait des règles de sélection évoquées dans la \autoref{sec:InterpretationCouplage}, cette condition est vérifiée si la polarisation $\Polarisation$ du \fl est purement linéaire ou purement circulaire. Nous nous plaçons dans ce contexte. On pourra consulter la référence~\cite{DeJ98} pour un exemple de calcul faisant intervenir les couplages entre \snx magnétiques.
}
%
L'expression~\nref{eq:EiModiff} se simplifie car $\VAL$ ne couple que certains états entre eux (voir la \autoref{sec:InterpretationCouplage}). 
\Resultat{
Ainsi, en développant l'expression~\nref{eq:EiModiff} grâce aux équations~\nref{eq:IlaserNphotons} à~\nref{eq:OpChamp}, et en supposant que le nombre de photons $\Nph \gg 1$, on aboutit à :
%
\begin{align}
	\Delta \Ener_i &= 
	\frac{e^2\,\Nph\,\hbar\,\pulsLaser}{2\,\epsilon_0\,\Vcav}
	\sum_{\substack{j \neq i \\ \Nph' = \Nph\pm1}} 
	\dfrac{
	\Module{\BraOpKet{A_j}{\Polarisation\cdot\OperR}{A_i}}^2
	}{\EneriN - \EnerjN}\nonumber\\
	&= 
	\frac{e^2\,I}{2\,\epsilon_0\,c\,\hbar}
	\sum_{j \neq i} 
	\Module{\BraOpKet{A_j}{\Polarisation\cdot\OperR}{A_i}}^2
	\,\left(\dfrac{1
	}{\Deltaij - \pulsLaser}
	+\dfrac{1
	}{\Deltaij + \pulsLaser}
	\right)
\pointformule
	\label{eq:DeltaEibis}
\end{align}
}
Les éléments de matrices $\BraOpKet{A_j}{\Polarisation\cdot\OperR}{A_i}$ peuvent se décomposer, en appliquant le \termetech{Théorème de Wigner-Eckart}, et prendre la forme suivante~\cite{CDL86b} : \Cahier{MécaQ CCT Tome 2 page 1057}
\newcommand{\LePhantom}{\vphantom{J'}}
\begin{equation}
	\BraOpKet{A_2}{\Polarisation\cdot\OperR}{A_1} 
	= %\frac{1}{\sqrt{2\,F_1+1}}	\,
	   \BraAbsKet{n_2,F_2}{\OperR}{n_1,F_1}
	\, \BraKet{F_2,1;m_2,\polarindice}{F_1,m_1} 
	\virguleformule
	\label{eq:ElementMatriceReduite}
\end{equation}
où :
\begin{ditemize}
	\item le terme $\BraAbsKet{n_2,F_2}{\OperR}{n_1,F_1}$ est l'\termetech{élément de matrice réduit} de l'opérateur $\OperR$, qui ne dépend que de la partie radiale des fonctions d'onde de $\Ket{A_1}$ et $\Ket{A_2}$, définie par les \termetech{nombres quantiques principaux} ($n_1, n_2$) et les \termetech{nombres quantiques azimutaux} ($F_1, F_2$),
\item le terme $\BraKet{F_2,1;m_2,\polarindice}{F_1,m_1}$ est un \termetech{coefficient de Clebsch-Gordan}, qui ne dépend que des nombres quantiques associés aux moments cinétiques ($F_1, F_2;m_1,m_2$) des états considérés et de la polarisation de la lumière laser, définie par $\polarindice$.
\end{ditemize}
En outre, comme nous l'avons mentionné dans la remarque \vpageref[en haute de cette page]{RQ:QuePolarLinEtCir}, l'équation~\nref{eq:DeltaEibis} n'est valable que dans les trois cas suivants :
\begin{itemize}
	\item lumière polarisée linéairement $\pi \rightarrow \polarindice = 0$,
	\item lumière polarisée circulairement $\sigma^+ \rightarrow \polarindice = +1$,
	\item lumière polarisée circulairement $\sigma^- \rightarrow \polarindice = -1$.
\end{itemize}

\subsubsection{Expression en fonction du taux d'émission spontanée $\pulsSpont$}
Le calcul de $\BraAbsKet{n_2,F_2}{\OperR}{n_1,F_1}$ fait intervenir des intégrales de recouvrement entre les fonctions d'onde des états considérés, mais il est possible de relier ce facteur au taux d'émission spontané $\pulsSpont$ entre deux orbitales dont les moments angulaires sont respectivement $J_1$ et $J_2$~\cite{Lou00} :
\begin{equation}
	\Module{\BraAbsKet{n_2,F_2}{\OperR}{n_1,F_1}}^2 = \pulsSpont
	\,\frac{3\,\epsilon_0\,\hbar\lambda^3}{8\,\pi^2\,e^2}
	\,\left( 2\,J_2+1 \right) \, \left( 2\,F_2+1 \right)
	\,\Module{\begin{Bmatrix}%
	J_1 & J_2 & 1 \\%
	F_2 & F_1 & I%
	\end{Bmatrix}}^2
\virguleformule
	\label{eq:GammaMatriceReduit}
\end{equation}
où le dernier terme est un \termetech{coefficient 6-$j$ de Wigner}, et $\pulsSpont$ est le taux d'émission stimulée associé à la transition dont la \lo est $\lambda$.

\casse

\subsection{Application au cas d'un atome alcalin}
Dans cette sous-section nous considérons plus spécifiquement 
%l'application de l'expression~\nref{eq:DeltaEibis} au 
le cas d'un atome alcalin dans son niveau fondamental ${\OrbitalSJ}$ (pour lequel $J_1=\tfrac{1}{2}$) et soumis à une onde laser.
Les états à considérer pour le calcul se résument alors à ceux mis en jeu dans les transitions des \emph{lignes $D_1$ et $D_2$}:
\[
\begin{cases}
\OrbitalSJ\xrightarrow[\text{ligne $D_1$}]{}\OrbitalPJun 
 \text{~~ ($J_2=\tfrac{1}{2}$)} 
 \\
\OrbitalSJ\xrightarrow[\text{ligne $D_2$}]{}\OrbitalPJdeux 
 \text{~~ ($J_2=\tfrac{3}{2}$)} 
\end{cases}
\]
Les \los respectives de ces deux lignes sont $\LO_1\approx\nm{795.0}$ et $\LO_2\approx\nm{780.2}$. 
On peut alors écrire l'expression~\nref{eq:DeltaEibis} :
\[
\begin{split}
	\Avec{\Delta \Ener_i}{i \in \OrbitalSJ} = \,\,\,\,\,
	\frac{e^2\,I}{2\,\epsilon_0\,c\,\hbar}\,
	&\left[
	\left(\dfrac{1
	}{\desacU}
	+\dfrac{1
	}{\desacU + 2\,\pulsLaser}
	\right)
	\,\sum_{j \in \OrbitalPJun} 
	\Module{\BraOpKet{A_j}{\Polarisation\cdot\OperR}{A_i}}^2 \right. \\
	&\,\,\,\,\,\,+ 
	\left.
	\left(\dfrac{1
	}{\desacD}
	+\dfrac{1
	}{\desacD + 2\,\pulsLaser}
	\right)
	\,\sum_{j \in \OrbitalPJdeux} 
	\Module{\BraOpKet{A_j}{\Polarisation\cdot\OperR}{A_i}}^2
	\right]
\virguleformule
\end{split}
\]
où les pulsations $\desacU$ et $\desacD$ correspondent aux désaccords de l'onde laser par rapport aux lignes $D_1$ et $D_2$ (nous avons implicitement supposé que ces désaccords sont grands devant les écarts énergétiques de la structure hyperfine des états excités, de manière à pouvoir sortir ces termes des sommations).

\Resultat{
En utilisant les propriétés des coefficients de Clebsch-Gordan et des coefficient 6-$j$ de Wigner, les expressions~\nref{eq:DeltaEibis},~\nref{eq:ElementMatriceReduite} et~\nref{eq:GammaMatriceReduit} permettent d'écrire l'expression ci-dessus sous une forme simplifiée faisant apparaître le moment angulaire total $F$ et le sous-état magnétique $\mF$ de l'état considéré:
\begin{equation}
\begin{split}
	\Avec{\Delta \Ener_i}{i \in \OrbitalSJ} = \,\,\,\,\,
	I\,\frac{\pulsSpont\,{\LO}^3}{16\,\pi^2\,c}\,
	&\left[
	\left(\dfrac{1
	}{\desacU}
	+\dfrac{1
	}{\desacU + 2\,\pulsLaser}
	\right)\,
	\left( 1 - \polarindice \, \gF \, \mF \vphantom{\dfrac{1}{\desacU + 2\,\pulsLaser}}
	\right)
	\vphantom{\,\sum_{j \in \OrbitalPJdeux} }
	\right. \\
	&\,\,\,\,\,\,+ 
	\left.
	\left(\dfrac{1
	}{\desacD}
	+\dfrac{1
	}{\desacD + 2\,\pulsLaser}
	\right)\,
	\left( 2 + \polarindice \, \gF \, \mF \vphantom{\dfrac{1}{\desacU + 2\,\pulsLaser}}
	\right)
	\vphantom{\,\sum_{j \in \OrbitalPJdeux} }
	\right]
\virguleformule
\end{split}
	\label{eq:DeltaEiFinalAlcalin}
\end{equation}
où $\gF$ est le \termetech{facteur de Landé}, $\polarindice$ est l'indice qui caractérise la polarisation de l'onde laser ($\polarindice=0,\pm1$ respectivement pour les cas linéaire $\pi$ et circulaire $\sigma^\pm$). Rappelons que le taux d'émission $\pulsSpont$ stimulée est associé à la transition dont la \lo est $\lambda$.
}

\ApplicationNumerique{
Évaluons l'expression~\nref{eq:DeltaEiFinalAlcalin} dans le cas du \Rb dans son état fondamental et en considérant l'interaction avec une onde polarisée \emph{linéairement} issue de notre \lyb ($\LO=\nm{1072}$).
On obtient alors : 
\[
	\Delta \Ener_i = \PropUI \times I 
	\virguleformule \nonumber \\
	\text{avec ~~} \PropUI \approx \SI{-2.06E-36}{\joule\per({\watt\per\square\meter})}
\pointformule
\]
Afin d'avoir, par exemple, un piège d'une profondeur correspondant à une température $T=\milliK{1}$, il faut une intensité d'environ $\Wpcmc{E6}$.
}

\subsection{Dépendance du potentiel en fonction de l'état interne}
L'expression~\nref{eq:UpropI} fait explicitement apparaître la dépendance du potentiel dipolaire en fonction du sous-état magnétique de l'atome. 
%Cet effet ne ce fait pas sentir que pour une polarisation linéaire ($q=0$).
Cette dépendance peut être mise  à profit dans certaines expériences comme, par exemple, pour mettre en \oe uvre des mécanismes de refroidissement de type \termetech{Sisyphe} dans un \pd~\cite{MDW02}. 

\ApplicationNumerique{
Pour évaluer l'importance relative de cette dépendance,
nous 
%pouvons reformuler l'expression~\nref{eq:UpropI} sous la forme :
%\[..................
%	\PropUI = \frac{\pulsSpont\,{\LO}^3}{16\,\pi^2\,c}
%	\, \left(\frac{2}{\desacD} + \frac{1}{\desacU} \right)
%	\, \left( 1 +
%	\frac{\desacU - \desacD}{2\,\desacU + \desacD}
%	\, \polarindice \, \gF \, \mF 
%	\right)
%\pointformule
%%	\label{eq:UpropIDependanceMF}
%\]
%Nous 
considérons l'expression~\nref{eq:UpropI} dans 
le cas du \Rb dans son état fondamental (pour lequel $\gF=\tfrac{1}{2}$), en interaction avec une onde polarisée \emph{circulairement} et issue de notre \lyb ($\LO=\nm{1072}$). 

La contribution du terme dépendant de $\mF$ correspond à une différence relative de potentiel de~~$\mF\!\!\times\!\val{2.2}\%$. On peut donc avoir une différence d'environ $\val{9}\%$ du potentiel entre les états $\EtatFmF{2}{2}$ et $\EtatFmF{2}{-2}$ du niveau fondamental.

Si on considère un piège dont la profondeur correspond à \microK{200}, l'écart énergétique entre deux sous-niveaux magnétiques voisins permet d'effectuer des transitions avec une onde \rf de typiquement \SI{100}{\kilo\hertz}.
}

