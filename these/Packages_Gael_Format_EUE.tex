\newcommand{\TitreThese}{
Manipulation et refroidissement par �vaporation forc�e d'ensembles atomiques ultra-froids pour la production d'un jet intense dans le r�gime de d�g�n�rescence quantique : vers l'obtention d'un \textit{laser � atomes continu}}

\usepackage{ifthen}
\usepackage{ifpdf}

%\usepackage[pdftex]{thumbpdf}
\usepackage[pdftex]{graphicx} % pour l'incusion de graphiques


\graphicspath{{images/}}
%\setkeys{Gin}{width=\linewidth}

\usepackage[intlimits]{amsmath}
\usepackage{amssymb}
% pour des repr�sentation de vecteur sympa pour les physiciens
\usepackage{vector}
\usepackage{array}
\usepackage{slashbox}% pour couper une case de tableau en deux

% figures avec sous-figures et sous-captions:
\usepackage[subrefformat=subparens]{subfig}

% pour des r�f�rences en language naturel
\usepackage[english]{varioref} 

% pour des supers graphiques trop cool, mieux que PSTricks
\usepackage{tikz,tkz-fct,tkz-2d}
\usetikzlibrary{shapes,snakes,arrows,patterns}

\usepackage[np,autolanguage]{numprint}

\usepackage[autoplay, loop, poster]{animate} % pour animer des dessins

% texte sur des images
\usepackage[abs]{overpic}
\setlength\unitlength{1cm}

%%%%%%%%%%%%%%%%%%%%%%%%%%%%%%%%%%%%%%%%%%%%%%%%%%%%%%%
% Pour pouvoir avoir des fleche depuis le texte vers des formule par exemple:
% http://www.fauskes.net/pgftikzexamples/global-nodes/
%%%%%%%%%%%%%%%%%%%%%%%%%%%%%%%%%%%%%%%%%%%%%%%%%%%%%%%
% For every picture that defines or uses external nodes, you'll have to apply the 'remember picture' style. To avoid some typing, we'll apply the style to all pictures.
\tikzstyle{every picture}+=[remember picture] 
% By default all math in TikZ nodes are set in inline mode. Change this to displaystyle so that we don't get small fractions.
\everymath{\displaystyle}

% espace int�ligent pour la fin d'une nexcommand en text
\usepackage{xspace}

%gestion des unit� !
\usepackage[amssymb]{SIunits}
\usepackage{sistyle}
\SIstyle{German} 

\usepackage{picins}
\usepackage{fancybox}

\usepackage{movie15}
% hachurer du texte avec \sout{Texte � barrer}\xout{Texte � hachurer}\uwave{Texte � souligner par une vaguelette}
\usepackage[normalem]{ulem}


% pour ajouter localement une ligne � une page
% contient une commande pour checker si on est sur un page odd or even
\usepackage{addlines}
\usepackage{mparhack}

% Si on veut des mini-tables des matieres (utiliser minitoc-hyper en conjonction avec tlhypref) :
\ifpdf % le monde PDF
\usepackage[french]{minitoc-hyper}
\else % le monde DVI/PS
\usepackage[french]{minitoc}
\fi
%\renewcommand{\mtctitle}{dfgh}

% Liste notations
\usepackage[refpage,english,intoc]{nomencl}  %intoc rajoute une ligne dans la toc

%\usepackage{tocbibind}                     % toc,lotf,bib,index in toc
%\usepackage[numbers,sort&compress]{natbib} % Good [47] type referencing
%\usepackage{hypernat}                      % Makes natbibs s&c play w/ hyperref

\ifpdf % le monde PDF
%\usepackage{hyperref} % pour des r�f�rences cliquables
\usepackage[pdftex, colorlinks = true
%						, pdfstartview = FitH
%						, colorlinks=false,
						, linkcolor = black, citecolor = black, urlcolor = black, pdfpagelabels, pagebackref]{tlhypref}
\hypersetup{pdfauthor={Gael Reinaudi},
pdftitle={\TitreThese},
pdfsubject={Th�se de doctorat},
pdfkeywords={physique,quantique,laser,atome,d�g�n�rescence,froid,manipulation,atomique,refroidissement,�vaporation,jet}}
\else % le monde DVI/PS
\fi
% ********************************************************************
% get the links to the figures and tables right
%\RequirePackage[all]{hypcap} % to be loaded after hyperref package
% ********************************************************************
% setup the style of the backrefs from the bibliography
\newcommand{\backrefnotcitedstring}{\relax}%(Not cited.)
\newcommand{\backrefcitedsinglestring}[1]{Cit� � la page~#1.}
\newcommand{\backrefcitedmultistring}[1]{Cit� aux pages~#1.}
\RequirePackage[hyperpageref]{backref} % to be loaded after hyperref package
   \renewcommand{\backreftwosep}{ et~} % seperate 2 pages
   \renewcommand{\backreflastsep}{, et~} % seperate last of longer list
   \renewcommand*{\backref}[1]{}  % Disable standard
   \renewcommand*{\backrefalt}[4]{% Detailed backref
      \ifcase #1 %
         \backrefnotcitedstring%
      \or
         \backrefcitedsinglestring{#2}%
      \else
         \backrefcitedmultistring{#2}%
      \fi} 
% ********************************************************************

% Si l'on produit le PDF avec pdflatex, ceci remplace la plupart
% des polices EC par des polices CM, plus adaptees a la generation de PDF,
% car ayant des equivalents PS :
\usepackage{aeguill}
% Pour tout savoir sur les polices
% (cette ligne n'est pas necessaire au traitement du fichier)
%\usepackage[infoshow]{tracefnt}

% �vite les under-over full box...
%\usepackage[auto=false]{microtype} % 2010-05-27
%\usepackage{microtype}

\usepackage[latin1]{inputenc}
\usepackage{times}
\usepackage[T1]{fontenc}
\usepackage{lmodern}



