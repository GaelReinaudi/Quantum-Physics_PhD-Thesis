
\DontFrameThisInToc
\chapter*{Pr�sentation G�n�rale}

Ca y est. C'est la question! je la connais bien celle la. De toute mani�re, � chaque fois (je dis bien � chaque fois) que je fais visiter le laboratoire � un ami, ou bien � un membre de la famille,... Je sais que \emph{la} question va arriver. Voil� pr�s d'une demi heure que j'explique qu'est-ce qu'un atome, les bases de sa strucutre, comment il se comporte, comment la lumi�re peut les pousser pour les acc�l�rer, ou bien les ralentir. Bien �videmment, la quait� du vide qui r�gne au coeur du dispositif exp�rimental n'impr�ssionne jamais les novices, car pour eux, le vide,... c'est vide! 

-"La pression dans cette enceinte est un million de millard de fois plus faible que la pression atmosph�rique!" 

-"Mais comment les hublot tiennent-ils?" Je souris et promet une r�ponse un peu plus tard.

-"Ce qu'il faut comprendre, c'est que meme � cette pression, il y a encore quelques dizaines de milliers d'atomes dans chaque centimetre cube." (information utile, mais completement contradictoire avec le fait que l'enceinte est vide...)

Les diodes laser n'arrive meme pas dans l'esprit de mes visiteurs jusqu'� �veiller le m�pris de ceux qui s'attendait a voir des \emph{vrai} lasers...

Et en fin la question, l'�ternelle question...

"Mais en fait, ca sert � quoi tout ca?"
Imm�diatement, avec l'habitude, et suivant l'interlocuteur, deux parades se pr�sente � moi. C'est d'ailleur souvent au fil de la pr�sentatio que j'ai d�j� choisi depuis longtemps laquelle fera sont effet.

Les voici:

\begin{enumerate}
	\item J'attaque la partie \emph{science et vie junior} du cerveau de l'agresseur en enchainant tout de suite sur les application possible (plausible,... imaginable) de notre exp�rience:
	\begin{itemize}
		\item les nanotechnologie (ses petites machines qui r�parent le corps humain de l'interieur)
		\item l'ordinateur quantique (aussi puissant sur le papier qu'une machine a remonter le temps)
		\item le GPS pr�cis au centimetre pr�s (sinonyme d'avion de ligne sans pilote,... ca impressionne toujours)
	\end{itemize} 
	\item J'attaque sur le cot� \emph{"Les nano-quoi?"} de la b�te et je d�cris l'apparition de deux d�couvertes profondement li�es � la m�canique quantique, et qui ont m�tamorphos� la plan�te au cour du XX�me si�cle:
	\begin{itemize}
		\item le laser, qui en 1961 �tait une curiosit� de la physique fondamentale.
		\item le transistor, sqfsfqs....
	\end{itemize} 
\end{enumerate}
