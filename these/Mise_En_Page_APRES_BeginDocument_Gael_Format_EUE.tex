%%%%%%%%%%%%%%%%%%%%%%%%%%%%%%%%%%%%%%%%%%%%%%%%%%%%%%%%%%%%%%%%%%%%%%%%%%%%%%%%%%%%%%%%%%%%%%%%%%%%%%%%%%%%%%%%%%%%%%%%%%%%%%
%%%%%%%%%%%%%%%%%%%%%%%%%%%%%%%%%%%%%%%%%%%%%%%%%%%%%%%%%%%%%%%%%%%%%%%%%%%%%%%%%%%%%%%%%%%%%%%%%%%%%%%%%%%%%%%%%%%%%%%%%%%%%%
%%%%%%%%%%%%%%%%%%%%%%%%%%%%%%%%%%%%%%%%%%%%%%%%%%%%%%%%%%%%%%%%%%%%%%%%%%%%%%%%%%%%%%%%%%%%%%%%%%%%%%%%%%%%%%%%%%%%%%%%%%%%%%

      \OddHead={{\leftmark\rightmark}{\hfil\slshape\rightmark\hfil}}
      \EvenHead={{\leftmark}{{\hfil\slshape\leftmark}\hfil}}
      \OddFoot={\hfil\thepage\hfil}
      \EvenFoot={\hfil\thepage\hfil}
      \pagestyle{Fancy}

%-------------------------------------------------------------------
%                          Encadrements
%-------------------------------------------------------------------

% encadre les chapitres dans la table des matieres:
% (ces commandes doivent figurer apres \begin{document}
\FrameChaptersInToc  
%\FramePartsInToc
\DontFramePartsInToc


%-------------------------------------------------------------------
%            Reinitialisation de la numerotation des chapitres
%-------------------------------------------------------------------
% Si la commande suivante est presente,
% elle doit figurer APRES \begin{document}
% et avant la premiere commande \part
%\ResetChaptersAtParts 

%-------------------------------------------------------------------
%               mini-tables des matieres par chapitre
%-------------------------------------------------------------------
% preparer les mini-tables des matieres par chapitre.
% (commande de minitoc.sty)
\dominitoc

%-------------------------------------------------------------------
%                  ecriture de `Chapitre' et `Partie' 
%                      dans la table des matieres
%-------------------------------------------------------------------
\WritePartLabelInToc
\WriteChapterLabelInToc


% ************************************************************************
% Pour la liste des notations
\ifthenelse{\FaitNomenclature > 0}{
    %Titre de la page
    \renewcommand{\nomname}{%\fontfamily{ppl}\selectfont %
    Principales notations utilis�es} 
    \ifthenelse{\FaitNomEnMarge > 0}{	
       \renewcommand{\nompreamble}{Nous d�finissons ici les principales notations utilis�es dans ce manuscrit, en pr�cisant le num�ro de page sur laquelle chacune d'entre elles appara�t pour la premi�re fois. De plus, afin de faciliter la lecture, chaque premi�re occurrence sera rep�r�e dans le manuscrit par une mise en marge de la notation. Un exemple de mise en marge est pr�sent� ici pour la notation $\flux(\Devi)$.\\
       ~\vspace{-1ex}\todo{\ensuremath{\flux(\Devi)}}%
       }       }{}%
    %pour avoir des lignes de ... entre la description et le #page
    \renewcommand{\pagedeclaration}[1]{~\dotfill\hyperpage{#1}}  
    %La commande \nomgroup permet de regrouper les symboles par groupe.
    \renewcommand{\nomgroup}[1]{%
      \ifthenelse{\equal{#1}{A}}{\medskip\item[]\item[]\hspace*{-\leftmargin}%
      \textbf{Ch.1 - \TitreChapitreUn }}{%
       \ifthenelse{\equal{#1}{B}}{\medskip\item[]\item[]\hspace*{-\leftmargin}%
       \textbf{Ch.2 - \TitreChapitreDeux}}{%
        \ifthenelse{\equal{#1}{C}}{\medskip\item[]\item[]\hspace*{-\leftmargin}%
        \textbf{Ch.3 - \TitreChapitreTrois}}{%
         \ifthenelse{\equal{#1}{D}}{\medskip\item[]\item[]\hspace*{-\leftmargin}%
         \textbf{Ch.4 - \TitreChapitreQuatre}}{%
          \ifthenelse{\equal{#1}{E}}{\medskip\item[]\item[]\hspace*{-\leftmargin}%
          \textbf{Ch.5 - \TitreChapitreCinq}}{%
           \ifthenelse{\equal{#1}{F}}{\medskip\item[]\item[]\hspace*{-\leftmargin}%
           \textbf{Ch.6 - \TitreChapitreSix}}{%
            }}}}}}%
%\medskip\item[]\hspace*{-\leftmargin}%
%\rule[2pt]{0.45\linewidth}{1pt}%
%\hfill #1\hfill
%\rule[2pt]{0.45\linewidth}{1pt}
        }
    %D�finit \Titre... comme une lettre qui correspond au num�ro du chapitre courant.
    \def\ChCourantTitre{\ifcase\value{chapter}\or A\or B\or C\or D\or E\or F\fi}  
    %La commande
\newcounter{nomereee}[chapter]
\stepcounter{nomereee}
\newcommand{\nome}[2]{\stepcounter{nomereee}\nomenclature[\ChCourantTitre\alph{nomereee}]{\ensuremath{#1}}{#2}%
    \ifthenelse{\FaitNomEnMarge > 0}{	\todo{\ensuremath{#1}}}{}%
    }
    \newcommand{\nomeRemonte}[3]{%
    \vspace{-#3}%
    	\nome{#1}{#2}%
    \vspace{#3}%
    }
    \makenomenclature
    %si FaitNomenclature=0, je red�finis la command \nome pour �viter les erreurs lors de la compilation
    }
    {\newcommand{\nome}[2]{}    \newcommand{\nomeRemonte}[3]{}} 



% ************************************************************************
% Pour fair une note dans la marge !
% Command for inserting a todo item
\definecolor{orange}{rgb}{1,1,0}
\tikzstyle{notestyleright} = [right, draw=black, fill=yellow]%, text width = 1cm]
\tikzstyle{notestyleleft} = [notestyleright, left]
\tikzstyle{connectstyle} = [draw = orange, thick]
\newcommand{\todo}[1]{%
% Add to todo list
%\addcontentsline{tdo}{todo}{\protect{#1}}%
%
\begin{tikzpicture}[remember picture]%, baseline=7.5ex]%
    \node [coordinate] (inText) {};
\end{tikzpicture}%
%
% Make the margin par
\marginpar[%
{% Draw note in left margin
    \tikz[remember picture] \path (0,0)++(2.2,0)++(-0.1,0)node[notestyleleft] (inNote) {#1};}%
]{% Draw note in right margin
    \tikz[remember picture] \path (0,0)++(0,0)node[notestyleright] (inNote) {#1};%
}}
